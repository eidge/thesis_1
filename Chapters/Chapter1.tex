% Chapter 1

\chapter{Introduction} % Main chapter title

\label{Chapter1} % For referencing the chapter elsewhere, use \ref{Chapter1} 

\lhead{Chapter 1. \emph{Introduction}} % This is for the header on each page - perhaps a shortened title

%----------------------------------------------------------------------------------------

\section{Motivation}

\subsection{Need}
Compared to 45 years ago we have an increase quality of life and consequently on life expectancy (6 to 8 years) and on aging population. 

(28 mil idosos vivem sozinhos ou isolados ~\cite{n_idosos} +NUMBERS) 

Limited health infraestructure to attend to their needs.

Among all this problems cardiovascular diseases are the number one worldwide cause of death.

\subsection{Approach}

To meet the need evidenced by this population medical devices sector has been developing (portable) solutions with integrated electronics that monitor vital parameters in the preferred living environment with the purpose of diagnosing diseases earlier and prevent emergency situations. These devices are part of the Ambient Assisted Living (AAL) area~\cite{jose}.

As said by the European Commision: "Ambient Assisted Living as a concept aims to prolongate the time people can live in a decent way in their own home by increasing their autonomy and self-confidence, the discharge of monotonously everyday activities, to monitor and care for the elderly or ill person, to enhance the security and to save resources."

It can be used in clinics, eldery houses, etc.

AAL concept leads to another one called Internet of Things (IoT). With IoT technology healthcare providers computer can be directly connected and constantly receiving data from all their patient's devices. The analysis of this data may lead to a more consider decision and make better use of the appointments.(more productive)

An ABI Research research says by 2020 there will be 30.000 million devices connected using this tecnology. ~\cite{iot}

The bigger medical devices players such as iHealth, Omron ISA and Exatronic already created some solutions that are being used by hypertenses. Specially with the ill population sensibility and accuracy is extremelly important

This project has as main goal the development hardware, firmware and software for a blood pressure measurement device with wireless communication of its readings.

What diseases does it cure?


\section{Contribution}

Exatronic

Modular medical device: oxymetry, electrotherapy, ultra-sounds, ECG

Nice engineering work

The Exa4Life is also very attentive to this market sector and the Ambient Assisted Living solutions. As such, you want to see the emergence of solutions with electronic able to monitor vital parameters in a simple way so they can be used by the target audience in their preferred environment.

It is intended, then the students to develop a prototype device which is a 7medidor blood pressure with wireless communication of their readings.

\section{Goals}

The main goals for this project are: Development of a blood pressure device prototype with wireless communication of its readings

Focus: hardware, firmware, software

design an electronic configuration that, given the relative performance vs. consumption, get a sign of heart Heart potential with good qualitative indices.

Com accuracy of X, dimmensions

Before/During - State of the Art, Electronic Configuration, System Tests

\begin{itemize}
\item \textbf {State of the Art Study} study of the blood pressure measurement devices, embedded systems and integrated circuits for medical area.
\item escolher os componentes: analisar as caraterísticas técnicas e a configuração e o desenho do hardware. Posteriormente, proceder-se-á
ao estudo das caraterísticas do sinal que se pretende medir e das normas existentes na conceção de dispositivos médicos. Os componentes a utilizar e a sua disposição irão ser detalhadamente estudados, e serão definidos em função dos objetivos delineados pela Exatronic para o produto a desenvolver.
\item \textbf {Design} the Printed Circuit Boards
  \item \textbf {Harware Development } Após esta parte inicial, e depois de definido o esquemático do módulo de aquisição,
haverá uma fase de aprendizagem de uma ferramenta CAD (software Altium
Designer®) e o desenvolvimento do circuito impresso (PCB, Printed Circuit Board).
  \item \textbf {Firmware Development} programming the microcontroller for proper operation/ C programming language
  \item \textbf{Software}Web APP bla bla bla
  \item \textbf{Software}Android APP bla bla bla
  \item \textbf{Test} the developed system that should be compared with a standard bla bla bla Por fim, serão feitos os testes à placa durante os quais será avaliada a sua funcionalidade e procurar-se-á otimizar o sinal obtido.
\end{itemize}

Figure 2 represents the tasks that had the ultimate goal of achieving the prototype generating plate signals.

\section{Thesis Outline}

This thesis is divided in 9 chapters.

\textbf {Chapter 1} defines the problem and the contribution of this Master Thesis to solve it, the main goals to be achieved at the end of the project and finally the report structure is presented for a better understanding of the topics covered during this work. 

\textbf {Chapter 2} portrays the stakeholders that made this project possible and illustrates the project scheduling.

\textbf {Chapter 3} details the physiolofical processes/most important concepts around blood pressure, its signal characteristics, the important components. Plus, it also points out the solutions refered in the litterature and the ones already in the market.

\textbf {Chapter 4: Acquisition System} refers

\textbf {Chapter 5: Hardware} describes

\textbf {Chapter 6: Firmware} presents

\textbf {Chapter 7: System Tests} makes sense of

\textbf {Chapter 8: Data Storage} specifies

\textbf {Chapter 9: Conclusion} recalls
