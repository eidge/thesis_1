% Chapter Template

\chapter{Blood Pressure Measurment} % Main chapter title

\label{Chapter3} % Change X to a consecutive number; for referencing this chapter elsewhere, use \ref{ChapterX}

\lhead{Chapter 3. \emph{Blood Pressure Measurment}} % Change X to a consecutive number; this is for the header on each page - perhaps a shortened title

%----------------------------------------------------------------------------------------
%	SECTION 1
%----------------------------------------------------------------------------------------

why blood pressure measurment is important? most used because

\section{Physiological Process}




Lorem ipsum dolor sit amet, consectetur adipiscing elit. Aliquam ultricies lacinia euismod. Nam tempus risus in dolor rhoncus in interdum enim tincidunt. Donec vel nunc neque. In condimentum ullamcorper quam non consequat. Fusce sagittis tempor feugiat. Fusce magna erat, molestie eu convallis ut, tempus sed arcu. Quisque molestie, ante a tincidunt ullamcorper, sapien enim dignissim lacus, in semper nibh erat lobortis purus. Integer dapibus ligula ac risus convallis pellentesque.

%-----------------------------------
%	SUBSECTION 1
%-----------------------------------
\section{Signal Characteristics}


Onda

Pressão sistolica

Pressão diastolica

Pressão média


Nunc posuere quam at lectus tristique eu ultrices augue venenatis. Vestibulum ante ipsum primis in faucibus orci luctus et ultrices posuere cubilia Curae; Aliquam erat volutpat. Vivamus sodales tortor eget quam adipiscing in vulputate ante ullamcorper. Sed eros ante, lacinia et sollicitudin et, aliquam sit amet augue. In hac habitasse platea dictumst.

\section{Components}

\textbf{Solenoid pneumatic valve}
external diammeter of 3 mm 

\textbf{Pressure Bomb}
external diammeter of 4.3 mm

\textbf{Patient Monitor NIBP connector}
external diammeter of 3,97 mm

\textbf{Pressure Sensor}
external diammeter of 3.04 mm

\textbf{T connector }
external diammeter of 3 mm

\textbf{Optocoupler}

\textbf{Tube}
external diammeter of 4 mm and internal diammeter of 2.5 mm

%-----------------------------------
%	SUBSECTION 2
%-----------------------------------

\section{State of the Art}

The former cenas whilst the latter cenas

Introduction to SoA (numbers)

%----------------------------------------------------------------------------------------
%	SECTION 2
%----------------------------------------------------------------------------------------

\subsection{Litterature}

For over a century the technique of blood pressure measurement developed by Riva-Rocci and Korotkoff has provided most of the data on hypertension diagnosis and treatment. Its limitations, however, are becoming increasingly evident and therefore alternative solutions are under investigation.~\cite{review}

\subsection{Solutions}

Braço

Fixo

Sensitivity etc

Normas de Segurança (Europeias, Americanas, etc)
\subsubsection{One Care Sensing}
One Care Sensing - http://www.onecare.pt/pt/pagina/2/

- ISA não avançou porque o financiamento parou (4 anos), apitava muito/má qualidade (FALAR COM A ISA)

O OneCare Sensing é um kit de fácil utilização que permite monitorizar a tensão arterial, frequência cardíaca, peso e glicemia, no domicílio do utilizador. Oferece ainda a possibilidade
de um prestador de cuidados ou profissionais de saúde acompanharem o estado de saúde do utilizador, à distância, contactando-o sempre que ocorram alterações relevantes nos parametros avaliados[24]. 

Medições são feitas pelo utilizador, no conforto da sua casa e ficam automaticamente disponíveis online no portal OneCare. Freq ajustável, comunicação bluetooth 

Envio de alertas quando ha desvios (registados no portal, SMS ou mail)

GLouzada

\subsubsection{iHealth}
In 2012 iHealth launched the first wireless blood pressure monitor, and in 2013, the first wireless blood glucose monitor. iHealth made the first health devices to be carried in Apple retail stores. Its products are now sold in shops like Walgreens, Best Buy, and Amazon.

25M Xiaomi

\subsubsection{Omron} 

Omron

\subsubsection{Plux BioSignals}
Pressão de Volume Sanguíneo (BVP do inglês Blood Volume Pressure): O sensor de Pressão de Volume Sanguíneo é um sensor ótico não-invasivo que mede variações do volume sanguíneo numa extremidade arterial, baseado na técnica de fotopletismografia. Este sensor possui uma sonda para ser colocada na ponta do dedo com uma fonte de luz vermelha e um fotodetetor. Estes dois componentes estão em modo de deteção de transmissão, e devido à sua configuração permitem assinalar as duas fases do ciclo cardíaco (sístole e diástole). A aplicação mais comum deste tipo de sensor é a medição da frequência cardíaca e da variabilidade de frequência cardíaca.
No entanto, pode ser utilizado para outro tipo de estudos, como a avaliação da




\subsubsection{Software}
iRythm Zio Patch
Convertis AVIVO Mobile Patient Management System
Toumaz SensiumVitals

\section{Overview}

Aplicações: Pacotes de turismo sénior, farmácias, clínicas, lares, IPSS
A table resuming everything